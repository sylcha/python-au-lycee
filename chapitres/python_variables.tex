% !TEX root = ../memo/python_memo.tex
% !TEX encoding = UTF-8 Unicode
% creation: 2015-09-16


\chapter{Variables}

Dans certains cas, on peut avoir besoin d'un nombre plusieurs fois dans un programme. Plutôt
que d'avoir à le retaper à chaque fois son expression, Python nous propose un moyen simple
de manipuler les nombres~: les \textit{variables}\index{Variable}.

\section{Notion de variable}
\begin{enumerate}
\item Une variable est basiquement une boîte avec une étiquette
(voir figure \ref{python:variables:figures:declaration}). Côté utilisateur~: c'est
l'étiquette, \cad{} le nom de la variable.
Côté ordinateur~: c'est la boîte, en fait une adresse mémoire.
On peut faire appel à la fonction \motcle{id()} pour voir cette adresse mémoire\index{Adresse mémoire}.
\item Différent \textit{types}\index{Type (variable)}~: entier, réel, chaîne de caractères, liste, etc.
La fonction \motcle{type()} renseigne sur le type d'une variable.
\item Typage automatique~: en fonction de ce que l'on met dans la boîte, Python sait ce qu'il a le
droit de faire avec.
\item Il faut respecter un certain nombre de règles pour nommer les variables~:
    \begin{enumerate}
    \item caractères ou chiffre ou tiret-bas (\_) seulement,
    \item le premier caractère est obligatoirement une lettre,
    \item sensible à la casse (\texttt{chaine} et \texttt{Chaine} sont deux variables différentes)
    \item certains mots sont dits \textit{réservés}~: ils désignent des commandes Python et ne
    pas être utilisés comme nom de variables. On peut les connaître depuis l'interpréteur
    directement~:
\begin{pythoncode}
>>> import keyword
>>> keyword.kwlist
['False', 'None', 'True', 'and', 'as', 'assert', 'break', 'class', 'continue', 'def', 'del', 'elif', 'else', 'except', 'finally', 'for', 'from', 'global', 'if', 'import', 'in', 'is', 'lambda', 'nonlocal', 'not', 'or', 'pass', 'raise', 'return', 'try', 'while', 'with', 'yield']
\end{pythoncode}
    \end{enumerate}
\end{enumerate}



\section{Affectations}
Avec Python, on déclare la variable en même temps qu'on lui attribue une valeur. En reprenant
l'analogie des boîtes, il n'existe pas de boîte vide~: cela n'a pas de sens de préparer des boîtes
à l'avance avec Python, on les crée au moment où on en a besoin.

\begin{figure}
\caption{Une variable est une boîte étiquettée.}\label{python:variables}
\centering
\begin{subfigure}[t]{0.3\textwidth}
	\centering
	\begin{tikzpicture}
	\node[boite fermee] (a) at (0,0) {\texttt{a}};
	\end{tikzpicture}
	\caption{Une variable \texttt{a}.}
	\label{python:variables:figures:declaration}
	\end{subfigure}
	\quad
	\begin{subfigure}[t]{0.3\textwidth}
	\centering
	\begin{tikzpicture}
	\node[boite ouverte] (b) at (0,0) {\texttt{b}};
	\node[above right=of b] (nb) {$5$};
	\draw[-latex'] (nb) to[out=180, in=90] ($(ofg)!0.7!(ofd)$);
	\end{tikzpicture}
	\caption{On affecte $5$ à la variable \texttt{b}~:\linebreak \texttt{b=5}}\label{python:variables:figures:affectation}
	\end{subfigure}
	\quad
	\begin{subfigure}[t]{0.3\textwidth}
	\centering
	\begin{tikzpicture}
	\node[boite ouverte] (c) at (0,0) {\texttt{c}};
	\node[above right=of c] (chaine) {\texttt{"John"}};
	\draw[-latex'] ($(ofg)!0.7!(ofd)$) to[out=90, in=180] (chaine);
	\end{tikzpicture}
	\caption{On utilise la variable \texttt{c}~:\linebreak \texttt{print(c)}}\label{python:variables:figures:utilisation}
	\end{subfigure}
\end{figure}

Pour \textit{affecter} (ou assigner) une valeur à une variable (\cad{} mettre une étiquette à une
boîte \textbf{et} y mettre une certaine valeur, ces deux opérations se faisant en même temps, voir
figure \ref{python:variables:figures:affectation}),
 on utilise l'opérateur \motcle{=}.
\begin{pythoncode}
>>> a = 4           # Affecter la valeur 4 à la variable 'a'
>>> id(a)           # Adresse mémoire de la variable 'a'
140443874558720
>>> type(a)         # Afficher le type de 'a'
<type 'int'>
>>> a               # Afficher la valeur de 'a' (ne marche pas dans un script)
4
>>> b = 5           # Affecter la valeur 5 à la variable 'b'
\end{pythoncode}
Si on utilise le même nom de variable plusieurs fois, c'est la dernière valeur qui est retenue,
Python ayant jeté l'ancienne et mis l'autre à la place (voir figure \ref{python:variables:figures:utilisation2})~:
\begin{pythoncode}
>>> a=4
>>> print(a)
4
>>> a=5
>>> print(a)
5
\end{pythoncode}
\begin{figure}[h]
\centering
\caption{La variable \texttt{a} contenait $4$ et on y stocke $5$~: le nombre $4$ est jeté}
\label{python:variables:figures:utilisation2}
	\begin{tikzpicture}
	\node[boite ouverte] (a) at (0,0) {\texttt{a}};
	\node[above left=of a] (new) {$5$};
	\node[above right=of a, cross out, draw=focusColor] (old) {$4$};
	\draw[-latex'] (new) to[out=0, in=90] ($(ofg)!0.6!(ofd)$);
	\draw[-latex'] ($(ofg)!0.7!(ofd)$) to[out=90, in=180] (old);
	\end{tikzpicture}
\end{figure}
On peut utiliser les variables dans des calculs {voir figure \ref{python:variables:figures:calculs}~:
Python les remplace par leur valeur. S'il
trouve une boîte marquée \texttt{a}, il l'ouvre et utilise ce qu'il trouve dedans à la place de
\texttt{a} dans l'expression.
\begin{pythoncode}
>>> a * b           # Calcul avec les variables 'a' et 'b'
20
\end{pythoncode}

\begin{exercice}
Stocker la valeur du rayon d'un cercle dans une variable \texttt{r}.

Utiliser cette variable \texttt{r} pour calculer la longueur du cercle et l'aire du disque
correspondant.
\end{exercice}

\begin{figure}[h]
\caption{Utilisation des variables \texttt{a} et \texttt{b} dans une opération
puis affectation du résultat dans une
variable \texttt{y}.\label{python:variables:figures:calculs}}
\centering
\begin{tikzpicture}
\clip (-1,-0.5) rectangle (13,3.6);
\node[boite ouverte] (a) at (0,0) {\texttt{a}};
\node[above right=of a] (nba) {$5$};
\draw[-latex'] ($(ofg)!0.7!(ofd)$) to[out=90, in=180]  (nba);
\node[boite ouverte, right=3cm of a] (b) {\texttt{b}};
\node[above left=of b.north] (nbb) {$7$};
\draw[-latex'] ($(ofg)!0.7!(ofd)$) to[out=90, in=0] (nbb);
\node (op) at ($(nba)!0.5!(nbb)$) {$\times$};
\node[boite ouverte, right=5cm of b] (y) {\texttt{y}};
\draw[-latex'] (op) to[out=60, in=125] ($(ofg)!0.7!(ofd)$);
\end{tikzpicture}
\end{figure}

\begin{remarque}
Pour afficher la valeur d'une variable, on peut se contenter de taper son nom dans l'interpréteur.
Toutefois, c'est une fonctionnalité de l'interpréteur qu'on ne retrouve pas dans un programme.
Il faut alors utiliser la fonction \motcle{print()}.
\begin{pythoncode}
>>> print(a)                # Afficher la valeur de la variable 'a'
4
\end{pythoncode}
On peut passer autant d'arguments qu'on souhaite
à la fonction \texttt{print()}, on les sépare par une virgule.
\begin{pythoncode}
>>> print("Score :", a)     # Utilisation de 'a' dans une phrase.
Score : 4
\end{pythoncode}
\end{remarque}
\begin{exercice}
Stocker \texttt{"John"} dans la variable \texttt{name} et $23$ dans la variable \texttt{age}.
À l'aide de ces deux variables et de la fonction \texttt{print()}, afficher la phrase
\texttt{"John a 23 ans"}.
\end{exercice}

\subsection{Mise à jour d'une variable}
Dans une affectation, Python évalue en premier le membre à droite du signe \motcle{=} et met
la valeur de ce membre dans la variable. Ainsi, on peut utiliser la variable elle-même dans le
membre de droite.

Par exemple, pour augmenter la variable \texttt{a} de $1$, on fera~:

\begin{pythoncode}
>>> a=5                # on affecte 5 à a
>>> print(a)
5
>>> a=a+1              # on augmente a de 1
>>> print(a)
6
>>> a+=1               # raccourci...
>>> print(a)
7
\end{pythoncode}

\begin{exercice}
Soit une variable \texttt{n}. On cherche à mettre à jour sa valeur. Montrer comment on peut~:
\begin{enumerate}
	\item Augmenter \texttt{n} de 5.
	\item Tripler \texttt{n}.
	\item Diviser \texttt{n} par $10$.
\end{enumerate}
Donner deux écritures différentes pour chaque question.
\end{exercice}


\subsection{Différence Variable -- Fonction}
On peut définir une \textit{expression} à partir de variables, toutefois si une des variables de
l'expression  change de valeur, l'expression n'est pas mise à jour. Il faudra utiliser le concept
de \textit{fonction} (voir section \ref{fonctions} page \pageref{fonctions}).
\begin{pythoncode}
>>> y = 3 * a - 2 * b      # Affecter une expression à la variable 'y'
>>> print(y)
2
>>> a = 2                  # Nouvelle valeur pour la variable 'a'
>>> print(y)                # La variable 'y' n'est pas mis a jour
2
\end{pythoncode}
Ci-dessus, on définit \texttt{y} en fonction de deux autres variables
\texttt{a}  et \texttt{b}. Il y aura dans \texttt{y} la valeur de l'expression au moment où elle
est définie. Si on change les valeurs de \texttt{a} ou \texttt{b} après, la valeur de \texttt{y}
n'est pas affectée.

\begin{remarque}
Attention à ça~:
\begin{pythoncode}
>>> a = 6
>>> b = a
>>> print(b)
6
>>> a = 7
>>> print(b)               # La variable 'b' n'est pas liée a 'a'
6
\end{pythoncode}
\end{remarque}

\subsection{Affectations multiples}
Python autorise les multiples affectations, elles sont parfois une manière élégante d'aborder
un problème. Si elles sont possibles, il faut les manipuler avec précaution car elles sont parfois
difficile à interpréter.

\begin{pythoncode}
>>> a, b = 7, 9       # Affecte 7 a la variable 'a' et 9 a 'b'
>>> print(a)
7
>>> print(b)
9
>>> x = y = 3         # Affecte 3 aux variables 'x' et 'y'
>>> print(x)
3
>>> print(y)
3
\end{pythoncode}

\begin{exercice}
Dans chacun des cas suivants, prévoir ce qui est affiché en sortie puis vérifier la réponse dans
    un interpréteur.
    \begin{multicols}{2}
\begin{pythonexemple}
>>> a = 11
>>> b = a - 7 ; a = a + 3
>>> a, b
\end{pythonexemple}

\begin{pythonexemple}
>>> a = 11
>>> a += 3 ; b = a - 7
>>> a, b
\end{pythonexemple}

\begin{pythonexemple}
>>> a = 11
>>> a, b = a + 3, a - 7
>>> a, b
\end{pythonexemple}

\begin{pythonexemple}
>>> a = 11
>>> b, a = a - 7, a + 3
>>> a, b
\end{pythonexemple}
    \end{multicols}
\end{exercice}

\begin{exercice}
Soit \texttt{a} et \texttt{b} deux variables ayant respectivement pour valeurs $10$ et $3$.

Montrer deux moyens d'échanger les valeurs de \texttt{a} et \texttt{b}.
\end{exercice}

\section{Conversions}
Quelquefois, on a besoin de convertir une variable d'un certain type dans un autre type,
pour pouvoir faire certaines opérations interdites avec un type mais autorisées avec un autre.
On utilise le nom du type comme une fonction.

\begin{pythoncode}
>>> a = "3"
>>> 2 + a          # Python ne sait pas faire ! Et il le dit !!!
Traceback (most recent call last):
  File " <stdin> ", line 1, in <module>
TypeError: unsupported operand type(s) for +: 'int' and 'str'
>>> 2 + int(a)
5
\end{pythoncode}
Si on essaie d'additionner un entier avec une chaîne de caractères, on obtient un message d'erreur
car Python ne sait pas faire. Il faut donc convertir la variable \texttt{a} qui est du type chaîne
de caractères en entier en utilisant la fonction \motcle{int()}.

Quelques fonctions de conversion utiles~: \motcle{float()} pour convertir un objet en décimal,
\motcle{str()} pour convertir un objet en chaîne de caractères, \motcle{hex()} et \motcle{oct()}
pour convertir un objet en nombre en base 16 ou 8 respectivement (l'objet produit est une chaîne),etc.

\begin{exercice}
Tenter de convertir en entier les chaînes de caractères suivantes~: \texttt{"78"}, \texttt{"7.8"},
 \texttt{"7*8"}, \texttt{"nombre"}.

 Interpréter les résultats.
\end{exercice}

\section{Scripts}
En dehors de l'interpréteur qui permet d'avoir le résultat d'une ligne de commande directement,
on peut consigner plusieurs lignes de commandes Python dans un même fichier : un \textit{script}.

Un \textit{script} est de manière plus générale un fichier dans lequel on a inscrit une liste
d'instructions dans un certain langage compréhensible par l'ordinateur. Plus spécifiquement,
les scripts écrit en Python utilisent l'extension \texttt{*.py} comme nom de fichier,
comme par exemple dans \texttt{test.py}.

\subsection{Scripts dans IDLE}
\begin{enumerate}
	\item Pour \textit{éditer} un script dans IDLE, on clique sur \textsf{Fichier\,>\,Nouveau} ou
on tape \touche{Ctrl}+\touche{N}. Une nouvelle fenêtre apparait alors. On remarque l'absence de prompt
(pas de chevrons \texttt{>\,>\,>}) qui indique que nous ne sommes plus dans l'interpréteur mais
dans un éditeur.
On saisit le texte et on tape \touche{Enter} à la fin de chaque ligne.

En règle générale, on fait commencer un script Python par cette ligne~:
\begin{pythoncode}
# -*- coding:utf8 -*-
\end{pythoncode}

Cette ligne \label{python:variables:encodage}indique à la machine quel est l'encodage utilisé
pour écrire le script, c'est-à-dire le type de représentation numérique de chaque caractère
\footnote{Pour plus d'informations sur le codage des caractères,
voir \url{http://fr.wikipedia.org/wiki/Codage_de_caract\%C3\%A8res}.}.

Même si elle est utile, cette ligne est optionnelle et n'est pas nécessaire à la bonne
exécution du script.

\item Pour \textit{sauvegarder} un script, on choisit \textsf{Fichier\,>\,Sauvegarder} ou on tape
\touche{Ctrl}+\touche{S}. On choisit alors le dossier de destination\footnote{Penser à sauvegarder
dans votre espace personnel pour retrouver votre script sur tout le réseau~!}.

\item Pour \textit{exécuter} un script python depuis IDLE, on tape la touche \touche{F5} ou on choisit
\textsf{Run\,>\,Run Module} et l'ordinateur exécute les instructions inscrites dans le script
les unes après les autres, dans l'ordre où elles ont été écrites. Le résultat s'affiche dans une
nouvelle fenêtre.
\end{enumerate}

\subsection{Interaction}
On peut depuis un programme demander à l'utilisateur saisir une séquence de caractères au clavier
et stocker la chaîne ainsi produite dans une variable. C'est la fonction \motcle{input()}.

\begin{multicols}{2}
\begin{pythonexemple}
a=input("nombre ? ")
print(a)
\end{pythonexemple}

\begin{result}
nombre ? 5
5
\end{result}
\end{multicols}

\begin{exercice}
Demander à l'utilisateur son nom et stocker le résultat dans \texttt{name}.
Afficher ensuite \texttt{"Bonjour <name> !"} où \texttt{<name>} sera remplacé par le contenu de la
variable \texttt{name} (par exemple si \texttt{name} contient \texttt{"John"}, on affichera
\texttt{"Bonjour John !"}).
\end{exercice}
\begin{remarque}
La fonction \texttt{input()} renvoie une chaîne de caractères. Il faudra convertir au besoin pour
faire un calcul.

\begin{pythonexemple}
a=input("nombre ? ")
print("Double :", 2*a)                # pas de convertion
print("Vrai double :", 2*int(a))      # convertion de 'a' en entier
\end{pythonexemple}

\begin{result}
nombre ? 5
Double : 55
Vrai double : 10
\end{result}
\end{remarque}

\begin{exercice}
Écrire un programme qui demande à l'utilisateur le rayon d'un sphère puis affiche son volume.
On pourra reprendre la fonction écrite à l'exercice \ref{python:fonctions:exos:sphere}.
\end{exercice}

\subsection{Commentaires}
Toute ligne d'un script ou toute fin de ligne qui commence par le caractère \motcle{\#} est ignoré
par Python. On appelle cette ligne un \textit{commentaire} car il sert à la personne qui écrit le
script à expliquer ce qu'elle a voulu faire.

 Lorsque le commentaire comprend plusieurs lignes, on peut le mettre entre deux triples double guillemets~:

\begin{pythoncode}
"""
Mon long commentaire sur
plusieurs lignes ici
"""

# Cette ligne est totalement ignorée par Python

print "Hello world ! " # après le dièse, c'est ignoré
\end{pythoncode}
\newpage{}

Il faut prendre l'habitude de commenter vos programmes. On pourra prendre le code qui suit en exemple~:
\begin{pythoncode}
def hypotenuse(cote1, cote2):
    """
    Calcule la longueur d'une hypoténuse connaissant
    les côtés de l'angle droit (passés en paramètres).
    """
    return math.sqrt((cote1)**2 + (cote2)**2)

#---------------------- Calcul de AC -----------------------
AB = 9      # Premier coté
BC = -3     # deuxième côté

# Afficher la longueur de longueur AC
print hypotenuse(AB, BC)
\end{pythoncode}
Il faut expliquer ce que stockent les variables, donner des éléments d'explication d'un algorithme,
dire ce que fait une fonction, etc. C'est important pour les autres qui lisent votre code mais aussi
 pour vous-même lorsque vous reprennez un script que vous avez écrit il y a longtemps~: on ne se
 souvient jamais de la manière dont on a traité un problème quand le code n'est pas récent.



 \begin{exercice}[Jour de la semaine]
 Écrire un programme qui demande à l'utilisateur une date (successivement le jour, le mois, l'année)
 et qui retourne le rang du jour dans la semaine auquel correspond cette date.

En prenant $m$ pour le mois, $d$ pour le jour et $y$ pour l'année (pour janvier, $m$ doit avoir la
valeur $1$, pour février, $m$ doit avoir la valeur $2$, etc.), les formules
\footnote{Dans ces formules, les divisions sont des divisions entières et $\mod$ est l'opération
 \texttt{\%} en Python donnant le reste de la division entière.} suivantes (convenant
pour le calendrier grégorien) donnent au final $d_0$ où $0$ correspond à dimanche,
 $1$ à lundi, etc.~:
\begin{align*}
y_0&=y-\frac{14-m}{12}\\
x&=y_0+\frac{y_0}4-\frac{y_0}{100}+\frac{y_0}{400}\\
m_0&=m+12\left(\frac{14-m}{12}\right)-2\\
d_0&=\left(d+x+\frac{31m_0}{12}\right)\mod\,7
\end{align*}

\paragraph{Exemple}
Quel est le jour de la semaine correspondant au 14 février 2000~?
\begin{align*}
y_0&=2000-1=1999\\
x&=1999+\frac{1999}4-\frac{1999}{100}+\frac{1999}{400}=2483\\
m_0&=2+12\times1-2=12\\
d_0&=\left(14+3483+\frac{31\times12}{12}\right)\mod7=2500\mod7=1
\end{align*}
La réponse est donnée par $d_0=1$ donc le 14 février 2000 était un lundi.
 \end{exercice}
\newpage{}
 \begin{exercice}[Distance parcourue par un objet] On souhaite écrire un programme qui calcule
 la distance parcourue par un objet dans certaines conditions.
 \begin{enumerate}
 	\item La distance parcourue en mètres après $t$ secondes par un objet lancé en ligne droite à la vitesse
 $v_0$ (en mètres par seconde) depuis une position initiale $x_0$ (en mètres) est donné par
 l'expression $x_0+v_0t+\dfrac{gt^2}2$ où $g$ est une constante (accélération de la pesanteur)
 égale à environ $ 9,806\,65$.

 Écrire un programme qui demande à l'utilisateur les trois valeurs $x_0$, $v_0$ et $t$ et calcule
 la distance parcourue correspondante.
 \item En réalité, l'accélération de la pesanteur $g$ dépend de la latitude $L$ (mesurée en radians)
  et de l'altitude $h$ (mesurée en mètre). La formule suivante donne une valeur approchée de la valeur
   de $g$ en fonction de la latitude et de l'altitude faible en regard du rayon terrestre~:
\[ g=9,780327\times \left (1+5,3024\times10^{-3}\times\sin^2(L)-5,8\times10^{-6}\times \sin^2(2\times L)-3,086\times10^{-7}\times h\right )\]
Modifier le programme précédent pour qu'il prenne en compte cette formule en demandant la latitude
$L$ en degrés
\footnote{Pour information, Paris est à une latitude égale à $48,85341\degres{}\,\text{N}$,
Varsovie $52,22977\degres{}\text{N}$ et San Francisco $37,77493\degres{}\text{N}$.}
et l'altitude $h$ à l'utilisateur.
 \end{enumerate}
 Penser à écrire des fonctions pour les différents calculs et aussi à utiliser les fonctions de
 convertion (\texttt{float()}, \texttt{math.degrees()}, etc.)
 \end{exercice}

 \begin{exercice}[Coordonnées polaires]
 Écrire un programme qui demande à l'utilisateur les coordonnées cartésiennes d'un point du plan
 (dans un repère orthonormé) puis qui affiche la conversion  en coordonnées polaires de ce même
 point (voir figure  ci-dessous).
\begin{multicols}{2}
\begin{center}% Repère avec base vectorielle
\footnotesize
\begin{tikzpicture}[scale=0.7]
\def\ech{1}
\def\Xmin{-0.8}
\def\Xmax{4.8}
\def\Ymin{-0.8}
\def\Ymax{3.8}
% grille
\draw[thin, LightGray] (\Xmin*\ech,\Ymin*\ech) grid[step=5*\ech mm] (\Xmax*\ech,\Ymax*\ech);
% axes
\draw[-latex'] (\Xmin*\ech,0) -- (\Xmax*\ech,0);
\draw[-latex'] (0,\Ymin*\ech) -- (0,\Ymax*\ech);
% base vectorielle
\draw[thick, -stealth'] (0,0) node[below left] {$O$}
	-- (\ech,0) node[below, midway] {$\overrightarrow{i}$};
\draw[thick, -stealth'] (0,0) -- (0,\ech) node[left, midway] {$\overrightarrow{j}$};
% graduations
% à faire !!!
\draw[Crimson, dashed] (4*\ech,0) -- (4*\ech,3*\ech) node[above right] {$M$} node[midway, right] {$y$}
	--(0,3*\ech) node[midway, above] {$x$};

\draw[DarkGreen] (0,0) -- (4*\ech,3*\ech) node[midway, above, sloped] {$\rho$};
\coordinate (M) at (4*\ech,3*\ech);
\coordinate (O) at (0,0);
\coordinate (I) at (\ech, 0);
\markangle[DarkGreen]{I}{O}{M}{$\theta$}{6};
\end{tikzpicture}
\end{center}
\columnbreak
Dans un repère $\Oij$, on peut repérer un point $M$ avec ses coordonnées cartésiennes $\coord Mxy$
ou avec des coordonnées polaires $\coord M{\rho}{\theta}$ où $\rho=OM$ et
$\theta=\left(\vect{i}, \ora{OM}\right)$ en radians.


\end{multicols}
\textbf{Indications}~:
On pourra utiliser une fonction pour calculer la distance entre deux points
connaissant leurs coordonnées cartésiennes ainsi que la fonction
\texttt{math.atan2(y,x)} qui retourne la valeur en radians comprise entre $-\pi$ et $\pi$ de l'angle
 $\left(\vect{i}, \ora{OM}\right)$ où $\coord Mxy$ dans $\Oij$.
 \end{exercice}