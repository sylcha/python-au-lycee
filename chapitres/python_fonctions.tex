% !TEX root = ../memo/python_memo.tex
% !TEX encoding = UTF-8 Unicode
% creation: 2015-09-10

\chapter{Débuter avec Python}

\section{Python comme machine à calculer}

\subsection{Environnement de travail}

Dans le menu \textsf{Démarrer}, taper \og{} IDLE \fg{} et choisir \textsf{IDLE Python} dans la liste des programmes
proposés. Vous venez d'ouvrir une console d'interprétation du langage Python. On va pouvoir parler
à l'ordinateur en utilisant ce langage et lui demander d'exécuter certaines tâches.
On saisi une phrase puis on tape sur entrée et l'interpréteur traduit la ligne de commandes
et fait le travail.


Commençons par des calculs.

\subsection{Opérations de bases}
Il existe plusieurs opérateurs pour faire les opérations courantes

\begin{pythoncode}
>>> 1 + 1                  # Addition
2
>>> 9 - 13                 # Soustraction
-4
>>> 2 * 3                  # Multiplication
6
>>> 17 / 4                 # Division
4.25
>>> 2 ** 4                 # Puissance
16
>>> (2 - 3) * 6 + 4 / 2    # Memes priorités qu'en maths
-4
\end{pythoncode}

Cas de la division entière (ou encore la division euclidienne)\index{Divisions (2 types)}.

\begin{pythoncode}
>>> 17 / 4       # Division décimale
4.25
>>> 17 // 4      # Division entière _explicite_
4
>>> 17,0 / 4     # ATTENTION : La virgule en Python est un point (notation,
(17, 0)          # américaine) d'où le résultat inattendu
\end{pythoncode}

On a une syntaxe pratique pour l'écriture scientifique et plus généralement pour les multiplications
par des puissances de $10$~:

\begin{pythoncode}
>>> 2e3      # 2 x 10^3
2000.0
>>> 2E3      # `E` majuscule marche aussi
2000.0
>>> 3e-3     # 3 x 10^-3
0.003
>>> 31.27e5  # 31,27 x 10^5
3127000.0
\end{pythoncode}

\begin{exercice}
Calculer les valeurs des expressions suivantes avec Python~:
\begin{align*}
A &= (4\times5 - 10)\div 3
&
B &= \frac{4}{3}+5\times8-3\\
C &= (3^2-1)(3-5^2)
&
D &= 2\times(-3)^2-4\times(-3)+7\\
E &= \left(4^6-12^3\right)\times(1-9^4)
&
F &= \frac{4\times10^{21}+12\times10^{23}}{8\times10^{25}}\\
\end{align*}
\end{exercice}

\begin{exercice}
Sachant que la vitesse de la lumière est de $299\,792\,458\,\text{m}.\text{s}^{-1}$,
donner le nombre de kilomètres parcouru par la lumière en un an (soit une année lumière).
\end{exercice}

\begin{exercice}
Utiliser Python pour calculer le reste de la division euclidienne de $2^{32}+1$ par 641. On pourra commencer par se débrouiller avec la division entière ("//"), puis on comparera avec l'opération "\%" existant déjà en Python.
\end{exercice}

\begin{exercice}\label{python:debuts:exofct}
Soit une fonction \[h\,:\,x\longmapsto 3x^2-5x+5\]
 Calculer à l'aide de Python les images	de $-2$, $1$ et $7$.
\end{exercice}

\subsection{Opérations plus complexes}
Pour calculer des opérations plus complexes comme des racines carrées par exemple, on doit dire à
Python de regarder dans un livre car il ne sait pas le faire tout seul. On doit faire appel à une
\textbf{librairie} qui va appporter plus de fonctionalité au langage.

Pour faire des calculs en mathématiques, on fait appel à la librairie\dots{} \motcle{math}~!
On utilise pour cela les mots clés \motcle{from} et \motcle{import}\index{Fonctions!Mathématiques}.

\begin{pythoncode}
>>> from math import sqrt      # 'sqrt' pour 'square root', la racine carrée
>>> sqrt(2)
1.4142135623730951
\end{pythoncode}
Dans l'exemple ci-dessus, on indique à Python qu'on souhaite utiliser la fonction \motcle{sqrt()} pour
calculer une racine carrée. Une fois cette ligne saisie, Python connaîtra cette fonction.

On peut aussi indiquer à Python d'importer \textit{toute} une librairie.

Il existe pour cela deux méthode. Dans la première, on charge \textit{seulement} la librairie
dans la mémoire de Python.Il faudra alors \textit{référencer}
les fonctions à la librairie importée. Par exemple, on dira \motcle{math.sqrt()} et non plus \motcle{sqrt()}.

\begin{pythoncode}
>>> import math              # on importe toute la librairie 'math'
>>> math.sqrt(2)             # on référence la fonction 'sqrt'
1.4142135623730951
>>> math.cos(2*math.pi)      # Python mesure les angles en radians !
1.0
>>> math.degrees(math.pi/4)  # une conversion en degrés d'un angle en radian
45.0
\end{pythoncode}


\begin{remarque}
Attention~: bien faire la différence entre une \textit{fonction} comme \motcle{math.cos()}
qui nécessite des parenthèses (on passe un nombre comme paramètre) et une \textit{constante}
comme \motcle{math.pi} qui ne nécessite pas de parenthèses.
\end{remarque}

Dans la deuxième technique d'utilisation d'une librairie, on dit à Python de charger en mémoire
toutes les fonctions et constantes de la  librairie en une fois. On se sert alors du charactère
\motcle{*} (appelé \textit{wild card}). Dans ce cas, on a plus à référencer la fonction à la librairie.

Cette dernière méthode d'importation d'une librairie est toutefois déconseillée~:
en effet, si on utilise plusieurs librairies et que deux fonctions portent
le même nom, Python utilisera comme librairie la dernière importée seulement.

\begin{pythoncode}
>>> from math import *
>>> cos(pi)
-1.0
\end{pythoncode}

\begin{exercice}
Calculer les expressions suivantes à l'aide de Python~:
\begin{align*}
G &= \cos\left(\frac{3\pi}{2}\right)
&
H &= \tan(45\degres)\\
I &= \sqrt{2\times5^2-9\times5+9}
&
J &= \sqrt{5\cos\left(\frac{\pi}{5}\right)-3\sin\left(-\frac{\pi}{5}\right)}
\end{align*}
\end{exercice}

\begin{exercice}
Parce que ces bases de numération ont joué un rôle dans l'histoire de l'informatique, Python dispose
 (dans sa version de base) de fonctions de conversion en binaire (base 2), en octale (base 8) ou en
 hexadécimal (base 16). Utiliser les fonctions "bin", "oct" et "hex" pour convertir les entiers 17
 et 65 en base 2, 8 ou 16. Que penser du format d'affichage des résultats~?
\end{exercice}



\section{Fonctions}\label{python:fonctions}
Il est possible dans Python de définir ses propres \textit{fonctions}\index{Fonctions!Definitions}. C'est un bout de code,
quelques lignes de commandes qui décrivent une série de tâches, que Python va mettre en mémoire
et que nous allons pouvoir réutiliser tant que la session est ouverte.

\subsection{Un premier exemple}
Pour définir une fonction on utlise le mot clé \motcle{def}, suivi du nom de la fonction
\footnote{Attention, il ne faut pas utiliser un mot déjà employé~!}
suivi d'un couple de parenthèses, suivi de deux points \motcle{:}
\footnote{Cette première ligne est la \textit{signature}\index{Fonctions!Signature}
de la fonction \texttt{resultat()}}.

Ensuite, il faut faire comprendre à Python quelles sont les lignes qui font partie de la fonction.
Pour cela, on utlise une {indentation}, c'est à dire un écart de $4$ espaces de la marge.

\begin{pythoncode}
>>> def resultat():            # 'signature' de la fonction
...    return 4**2 - 3          # bloc de code (notez l'indentation)
...
>>> resultat()                  # appel de la fonction
13
\end{pythoncode}

Ci-dessus, aux deux premières lignes, nous avons écrit une fonction \texttt{resultat()}
qui retourne (utilisation de la commande \motcle{return}) le résultat d'un calcul~: $4^2-3$.
Pour appeler la fonction, on saisit juste son nom
suivi des parenthèses. Attention à ne pas oublier les parenthèses par contre, sinon le résultat
est inattendu~!

\begin{pythoncode}
>>> resultat
<function resultat at 0x7f1a59c4aea0>
\end{pythoncode}

On peut utiliser cette fonction dans un autre calcul~:

\begin{pythoncode}
>>> 2 * resultat()
26
\end{pythoncode}

Par contre, notre fonction n'est pas très intéressante car elle fait toujours le même calcul.
Comment pourrait-on faire pour qu'elle calcule $x^2-3$ pour une valeur de $x$ que nous aurions
choisi~? C'est ce que nous allons voir dans un deuxième exemple.

\subsection{Notion de paramètre}
Lorsque nous avons utilisé la fonction \motcle{math.sqrt()} pour calculer $\sqrt{2}$, nous avons
passé le nombre $2$ comme \textit{argument} à la fonction~:

\begin{pythoncode}
>>> math.sqrt(2)              # on donne '2' comme argument à la fonction 'sqrt()'
1.4142135623730951
\end{pythoncode}

On va donc utiliser le même principe dans la fonction qui doit calculer $x^2-3$.

\begin{pythoncode}
>>> def f(x):                 # définition du paramètre 'x' avec la fonction
...     return x**2 - 3        # utilisation du paramètre 'x' dans le calcul
...
\end{pythoncode}

Entre les parenthèses dans la signature de la fonction, on introduit un \textit{paramètre}
\index{Fonctions!Paramètres} nommé
ici \texttt{x} (mais nous aurions pu lui donner un tout autre nom comme \texttt{choucroute} mais
c'est plus long). Cela indique à Python que lorsque nous allons appeler la fonction, il faudra lui
passer un argument, en l'occurence un nombre.

\begin{pythoncode}
>>> f(2)                      # appel de la fonction avec l'argument '2'
1
>>> f(4)                      # appel de la fonction avec l'argument '4'
13
\end{pythoncode}

Par exemple, lorsque nous appelons la fonction avec l'argument $2$, Python associe le nombre $2$
au paramètre \texttt{x}. Lorsqu'on réutilise le paramètre dans l'expression du calcul,
Python remplace $x$ par $2$.

Si nous appelons la fonction \texttt{f} sans argument (comme la fonction \motcle{resultat()}), on a
un message d'erreur~: Python nous dit qu'il ne sait pas quoi faire parce qu'il attend un argument~!

\begin{pythoncode}
>>> f()
Traceback (most recent call last):
  File " <stdin> ", line 1, in <module>
TypeError: f() missing 1 required positional argument: 'x'
\end{pythoncode}

\begin{exercice}
Écrire une fonction qui calcule l'expression $(x-3)(2x+9)$ pour une valeur de $x$ choisie par l'utilisateur.
\end{exercice}

\begin{exercice}
Reprendre l'exercice \ref{python:debuts:exofct} mais le traiter à l'aide d'une fonction.
\end{exercice}

\begin{exercice}
Écrire les fonctions qui calculent les grandeurs suivantes~:
\begin{enumerate}
	\item l'aire d'un carré connaissant la longueur de son côté~;
	\item le périmètre d'un cercle connaissant son rayon (utiliser la constante Python
	\motcle{math.pi}).
\end{enumerate}
\end{exercice}

\begin{exercice}\label{python:fonctions:exos:sphere}
Appel d'une fonction depuis une autre fonction.
\begin{enumerate}
	\item Écrire une fonction \texttt{cube()} qui renvoie le cube de son argument.
	\item Écrire une fonction \texttt{volume\_sphere()} qui renvoie le volume d'une shpère
	de rayon $r$ passé en argument. Cette fonction devra utiliser la fonction \texttt{cube()}
	précédente ainsi que la constante \texttt{math.pi}.
\end{enumerate}
\end{exercice}

\subsection{Plusieurs paramètres}
On peut utiliser le même concept avec plusieurs paramètres. Il suffit de séparer par une
virgule \motcle{,} les paramètres  dans la signature et les arguments dans l'appel de la fonction~:

\begin{pythoncode}
>>> def hypotenuse(a, b):
...     return math.sqrt(a**2 + b**2)
...
>>> hypotenuse(3, 4)
5.0
\end{pythoncode}

La fonction \texttt{hypotenuse()} utilise deux arguments \texttt{a} et \texttt{b}.
Au fait, que fait cette fonction~?

\begin{exercice}
Écrire une fonction \texttt{aire()} qui prend en paramètres la longueur et la largeur
d'un rectangle et qui renvoie son aire.
\end{exercice}

\begin{exercice}
Écrire une fonction \texttt{volume\_boite()} qui calcule le volume d'une boîte connaissant
la  largeur \texttt{l}, la hauteur \texttt{h} et la profondeur \texttt{p}
qui seront les paramètres de la fonction.
\end{exercice}

\begin{exercice}[Conversion de degrés]
Si $F$ est une température en degré Fahrenheit, la température en degré Celsius est donnée par
$\dfrac59(F - 32)$.

Écrire une fonction \texttt{convert\_F\_to\_C()} qui convertit une température donnée en degrés
Fahrenheit en degrés Celsius.
\end{exercice}

\begin{exercice} [Triangle]
\begin{enumerate}
	\item Écrire une fonction \texttt{perimetre\_triangle()} qui prenne en paramètres la longueur de
	ses trois côtés et qui renvoie la valeur de son périmètre.
	\item Écrire une fonction \texttt{aire\_triangle()} qui prenne en paramètres la longueur de
	ses trois côtés et qui renvoie la valeur de son aire.

	On donne la formule suivante\footnote{que l'on doit à Héron d'Alexandrie, un mathématicien grec
	du I\ier{} siècle après J-C.}
	qui permet de calculer l'aire $\mathcal{A}$ connaissant les longueurs $a$, $b$ et $c$ des trois
	côtés~:

	\[\mathcal{A} = \sqrt{p(p-a)(p-b)(p-c)}\quad\text{avec}\quad p = \frac 12 (a+b+c)\]
\end{enumerate}
\end{exercice}