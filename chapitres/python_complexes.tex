% !TEX root = ../memo/python_memo.tex
% !TEX encoding = UTF-8 Unicode
% creation: 2015-10-09

\chapter{Nombres complexes}

Python sait calculer avec des nombres complexes. Nous allons voir dans cette partie comment.

\section{Implémentation native}\label{python:complexes:natif}

Sans librairie particulière, Python connait les nombres complexes et sait calculer avec.

\subsection{Définition}\index{Complexes!Définition}
On utilise la lettre \motcle{j} dans la définition du nombre à la place du $i$ utilisé
en mathématiques\footnote{Les physiciens font de même.}. Par exemple, le nombre $3i$ s'ecrira
\texttt{3j}.

\begin{pythoncode}
>>> a = 3j
>>> type(a)
<class 'complex'>
\end{pythoncode}

\begin{remarque}
Attention~! \texttt{j} ne s'utilise pas tout seul. En effet, il pourrait être confondu avec
la variable du même nom.
\begin{pythoncode}
>>> b = 3 + j
Traceback (most recent call last):
  File " <stdin> ", line 1, in <module>
NameError: name 'j' is not defined
>>> b = 3 + 1j
\end{pythoncode}
\end{remarque}


\subsection{Parties réelles et imaginaires}
On peut retoruver les parties réelles \index{Complexes!Réelle, Partie} et imaginaires
\index{Complexes!Imaginaire, Partie} d'un nombre
complexe~: Python les conçoit comme
des \textit{attributs} du nombre complexe créé et les nomme \motcle{real} et \motcle{imag}~:
\begin{pythoncode}
>>> a = -2 + 5j
>>> a.real
-2.0
>>> a.imag
5.0
\end{pythoncode}
On peut aussi déterminer le \textit{conjugué}\index{Complexes!Conjugué} d'un nombre complexe à l'aide
de la fonction \motcle{conjugate()}~:
\begin{pythoncode}
>>> a = -2 + 5j
>>> a.conjugate()
(-2-5j)
\end{pythoncode}

Pour déterminer le \textit{module} d'un nombre complexe\index{Complexes!Module} on utilise la fonction
\motcle{abs()}~:

\begin{pythoncode}
>>> a = 1 + 1j
>>> abs(a)
1.4142135623730951
\end{pythoncode}

\begin{exercice}
Vérifier sur des exemples que si $u=a +ib$ alors $|u|=\sqrt{a^2+b^2}$.
\end{exercice}


\subsection{Opérations}
On peut effectuer les quatre opérations courantes~:

\begin{pythoncode}
>>> a = 3 - 2j
>>> b = 3 + 1j
>>> a + b
(6-1j)
>>> a - b
-3j
>>> a * b
(11-3j)
>>> a / b
(0.7000000000000001-0.8999999999999999j)
\end{pythoncode}

\begin{remarque}
Python effectue les conversions si besoin est~:
\begin{pythoncode}
>>> a = 3 - 2j
>>> c = 4
>>> type(a) ; type (c)
<class 'complex'>
<class 'int'>
>>> a + c                # Pas d'erreur !
(7-2j)
\end{pythoncode}
\end{remarque}

\begin{exercice}
Vérifier sur des exemples que si $u=a+ib$ et $v=c + id$ alors on a les formules suivantes~:
\begin{align*}
uv &= (ac - bd) + i(bc+ad)
&
\frac{u}{v} &= \frac{ac +bd}{c^2 + d^2} + i\frac{bc-ad}{c^2+d^2}
\end{align*}
\end{exercice}

On peut aussi calculer des puissances avec ces nombres complexes~:
\begin{pythoncode}
>>> a = 3 - 2j
>>> a ** 2
(5-12j)
\end{pythoncode}
On peut tenter de trouver la racine carrée d'un puissance utilisant l'opérateur \texttt{**} mais
on n'obtient qu'une des deux racines carrées~:
\begin{pythoncode}
>>> a ** 0.5
(1.8173540210239707-0.5502505227003375j)
\end{pythoncode}


\section{Librairie \texttt{cmath}}
L'utilisation de la librairie \motcle{cmath} est moins intuitive mais permet des calculs plus\dots\
complexes~! On va surtout pouvoir accéder à la forme trigonométrique des nombres complexes.

On charge la librairie \texttt{cmath} comme une librairie usuelle
\footnote{elle est présente par défaut, par besoin d'installation supplémentaire}~:

\begin{pythoncode}
>>> import cmath
\end{pythoncode}

\subsection{Forme trigonométrique}

Lorsqu'on a saisi un nombre complexe à l'aide de ce que nous avons vu section
\ref{python:complexes:natif}  comme par exemple \texttt{a = 1 + 1j}, alors on va pouvoir grâce à la
fonction \motcle{polar()} accéder à sa forme trigonométrique~: cette fonction renvoie
le \textit{module} et l'\textit{argument} (en radian) correspondant.

\begin{pythoncode}
>>> a = 1 + 1j
>>> cmath.polar(a)
(1.4142135623730951, 0.7853981633974483)
\end{pythoncode}

On peut avoir accès directement au module et à l'argument à l'aide des fonctions
\texttt{abs()} (déjà vu ci-dessus) et \motcle{phase()} \index{Complexes!Argument}~:

\begin{pythoncode}
>>> abs(a)
1.4142135623730951                  # module
>>> cmath.phase(a)
0.7853981633974483                  # argument (en radians)
>>> import math
>>> math.degrees(cmath.phase(a))    # pour avoir l'argument en degrés
45.0
\end{pythoncode}

On peut définir un nombre complexe à partir de son argument et de son module à l'aide de la fonction
\motcle{rect()}.

\begin{pythoncode}
>>> cmath.rect(4, math.pi/3)
(2.0000000000000004+3.4641016151377544j)
\end{pythoncode}

\subsection{Fonctions spécifiques}

Le module \texttt{cmath} dispose d'un fonction racine carrée \motcle{sqrt()} plus évoluée que le module
\texttt{math}.

\begin{pythoncode}
>>> math.sqrt(-3)
Traceback (most recent call last):
  File " <stdin> ", line 1, in <module>
ValueError: math domain error                  # calcul impossible avec 'math'
>>> cmath.sqrt(-3)
1.7320508075688772j                            # pas de problème avec 'cmath'
\end{pythoncode}

Le module \texttt{cmath} dispose d'une exponentielle complexe. Un exemple avec $\E^{\pi i}=-1$~:
\begin{pythoncode}
>>> cmath.exp(cmath.pi*1j)
(-1+1.2246467991473532e-16j)       # correct... à une approximation près !
\end{pythoncode}

On voit qu'on a un problème d'approximation dans les calculs. Le module \texttt{cmath} possède
une fonction \motcle{isclose()} qui renvoie \texttt{True} si deux nombres sont proches l'un
de l'autre. Cela est bien utilise pour les tests. On peut règler la tolérance à l'aide d'un argument
supplémentaire (voir la documentation
\footnote{\url{https://docs.python.org/3.5/library/cmath.html\#classification-functions}}
pour plus d'informations).

\begin{pythoncode}
>>> a = cmath.exp(cmath.pi*1j)
>>> cmath.isclose(a, -1)
True
>>> a == -1
False
\end{pythoncode}

\begin{exercice}
Vérifier sur des exemples que $\E^{ix}=\cos x + i \sin x$.
\end{exercice}